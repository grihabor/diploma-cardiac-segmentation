\section{Описание практической части}
Предложенный алгоритм для оценки уровня усталости от просмотра определенного стереофильма работает по схеме, изображенной на Рис.
\begin{figure}[H]
\end{figure}

\subsection{Генерация стереовидео для экспериментов}
Для подготовки тестовых видеопоследовательностей в стереоформате искусственным образом были добавлены искажения в исходные стереопары. Для преобразования ракурсов стереовидео использовался язык AviSynth \cite{avisynthfaqavisynth}, средствами которого можно осуществить необходимые преобразования: поворот исходного видео; изменение масштаба; изменение частоты кадров, с помощью которого моделируется сдвиг во времени; добавление цветового искажения (с помощью модуля, разработанного видеогруппой лаборатории <<Компьютерной графики и мультимедиа>>). Так как итоговое стереовидео содержит большое количество сцен с разными искажениями, программа на языке AviSynth состоит из множества шаблонных частей, для генерации которых была разработана программа на языке Python.

\subsection{Анализ качества стереовидео с помощью программ проекта Video Quality Measurement Tool 3D}
Для оценки качества стереовидео использовались программные продукты, разработанные видеогруппой лаборатории <<Компьютерной графики и мультимедиа>> в рамках проекта VQMT3D \cite{vqmt3d}. Стереовидео, которые использовались для проведения эксперимента, были проанализированы алгоритмами оценки:
\begin{itemize}
	\item{положительной диспаратности - максимальное смещение объектов заднего плана дальше от зрителя относительно плоскости экрана}
	\item{отрицательной диспаратности - минимальное смещение объектов переднего плана ближе к зрителю относительно плоскости экрана}
	\item{цветового несоответствия ракурсов стереопары}
	\item{поворота одного из ракурсов стереопары относительно другого}
	\item{несоответствия масштаба ракурсов стереопары}
	\item{несоответствия между ракурсами стереопары по времени}
	\item{смещения по вертикали одного из ракурсов стереопары относительно другого}
	\item{трапециедального искажения, вызванного съемкой с конвергированными осями камер}
	\item{несоответствия ракурсов стереопары по резкости}
	\item{средней яркости кадра}
	\item{стандартного отклонения яркости в кадре}
	\item{средней диспаратности в кадре}
	\item{среднего вектора движения в кадре}
	\item{средниего вектора движения по вертикали в кадре}
\end{itemize}

\subsection{Обучение и тестирование моделей}
