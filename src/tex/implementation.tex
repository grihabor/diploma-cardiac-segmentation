\section{Описание практической части}

Проект был~реализован на~языке Python~3. Для реализации моделей использовался фреймворк tensorflow~\cite{tensorflow} и~библиотекa keras~\cite{keras}. Наборы данных были получены в формате DICOM, поэтому для распаковки данных использовалась соответствующая библиотека pydicom~\cite{pydicom}. Обработка данных в памяти осуществлялось с помощью библиотеки numpy~\cite{numpy}. Для работы с изображениями использовалась библиотека OpenCV~\cite{opencv}. Аугментация данных реализована с помощью встроенных в~keras средств для генерации и преобразования изображений налету \texttt{ImageDataGenerator}. Для всех моделей использовалось одинаковое количество эпох. Обучение и тестирование было реализовано с помощью обертки из keras, которая позволяет использовать scikit-learn API с унифицированным интерфейсом для обучения, применения и проверки точности модели.

Код структурирован в несколько модулей. Часть из них предназначена для чтения данных и приведения их к нормальному виду, как уже было описано ранее. Для каждого набора данных необходимо было немного менять код, чтобы на выходе получались данные в одном формате и одного размера для всех наборов данных.

Код обучения и тестирования модели один и тот же для всех наборов данных и для всех моделей, поэтому он вынесен в отдельный модуль. Для~реализации этих функций использовалась библиотека scikit-learn~\cite{sklearn}, а точнее пакет \texttt{sklearn.model\_selection}.

Каждая модель реализована в отдельном модуле и имеет унифицированный интерфейс. Для использования модели из соответствующего модуля необходимо импортировать функцию-getter, которая возвращает модель. Параметры модели можно задать при вызове функции. Модель представляет собой объект типа \texttt{keras.models.Model}, который построен с помощью Keras Functional API. Сначала задается каждый слой сети со своими параметрами, затем устанавливаются зависимости между слоями. Получившийся граф имеет несколько входов и выходов, на его основе и~создается объект модели. 

После создания модель уже можно использовать для предсказаний. Чтобы использовать модель для обучения, необходимо задать еще несколько параметров: оптимизатор, функцию ошибки, метрики для отслеживания качества модели. В качестве функции ошибки использовалось отрицание индекса Дайса~\eqref{eq:dice_index}, которого нет в библиотеке keras, поэтому оно было реализовывано вручную.

Архитектура выбранной модели похожа на U-Net с тем отличием, 
что в~каждом сверточном блоке второй сверточный слой с~растяжением~2,
как показано на рисунке~\ref{fig:dilated-unet}.

\begin{figure}[ht]
  \includegraphics[width=\textwidth,keepratio]{img/dilated-unet}
  \caption{Архитектура U-Net c~растянутой сверткой. Фиолетовым выделены сверточные слои со~значением растяжения~2.}
  \label{fig:dilated-unet}
\end{figure}


Нейронная сеть обучалась в течение 10 эпох как и другие сети, 
учавствовавшие в~сравнении. Среднее время обработки одной фотографии
состовляет примерно 290~мс.

\newpage
\subsection{Результаты} 

\begin{table}[h]
  \begin{center}
    \caption{Результаты на наборе данных LVSC} \label{tab:lvsc_results}
    \begin{tabular}{ |c||*{3}{c|} }
      \hline
      \multirow{2}{*}{Метод}      & Количество   & \multicolumn{2}{c|}{Индекс Дайса} \\ \cline{3-4}
                                  & параметров   & Endocardium   & Epicardium        \\ \hline
      \hline
      FCN                         & $\sim11$~млн & 0.90          & 0.89              \\ \hline
      U-Net                       & $\sim31$~млн & 0.92          & \textbf{0.90}     \\ \hline
      GridNet                     &  $\sim8$~млн & \textbf{0.93} & \textbf{0.90}     \\ \hline
      U-Net с~растянутой сверткой & $\sim31$~млн & 0.92          & 0.88              \\ 
      \hline
    \end{tabular}

    \vspace{0.6cm}
    
    \caption{Результаты на наборе данных RVSC} \label{tab:rvsc_results}
    \begin{tabular}{ |c||*{3}{c|} }
      \hline
      \multirow{2}{*}{Метод}      & Количество   & \multicolumn{2}{c|}{Индекс Дайса} \\ \cline{3-4}
                                  & параметров   & Endocardium   & Epicardium        \\ \hline
      \hline
      FCN                         & $\sim11$~млн & 0.84          & \textbf{0.86}     \\ \hline
      U-Net                       & $\sim31$~млн & 0.79          & 0.77              \\ \hline
      GridNet                     &  $\sim8$~млн & 0.82          & 0.81              \\ \hline
      U-Net с~растянутой сверткой & $\sim31$~млн & \textbf{0.85} & 0.83              \\ 
      \hline
    \end{tabular}

    \vspace{0.6cm}
    
    \caption{Результаты на наборе данных Sunnybrook} \label{tab:sunnybrook_results}
    \begin{tabular}{ |c||*{3}{c|} }
      \hline
      \multirow{2}{*}{Метод}      & Количество   & \multicolumn{2}{c|}{Индекс Дайса} \\ \cline{3-4}
                                  & параметров   & Endocardium   & Epicardium        \\ \hline
      \hline
      FCN                         & $\sim11$~млн & 0.92          & \textbf{0.96}     \\ \hline
      U-Net                       & $\sim31$~млн & \textbf{0.93} & 0.94              \\ \hline
      GridNet                     &  $\sim8$~млн & 0.90          & 0.93              \\ \hline
      U-Net с~растянутой сверткой & $\sim31$~млн & 0.91          & \textbf{0.96}     \\ 
      \hline
    \end{tabular}
  \end{center}
\end{table}

Лучший результат сеть показыавет на~наборе данных~RVSC, содержащем снимки правого желудочка, 
обгоняя~FCN по~качеству сегментации эндокарда. На~наборе Sunnybrook сеть показывает результат
лучше, чем на~наборе данных~LVSC, содержащем больше случаев с~самыми крайними снимками МРТ сердца,
на~которых наиболее сложно решить задачу сегментации.
