\section{Описание практической части}

Проект был~реализован на~языке Python~3. Для реализации моделей использовался фреймворк tensorflow и~библиотекa keras. Наборы данных были получены в формате DICOM, поэтому для распаковки данных использовалась соответствующая библиотека dicom. Обработка данных в памяти осуществлялось с помощью библиотеки numpy. Для работы с изображениями использовалась библиотека OpenCV. Аугментация данных реализована с помощью встроенных в~keras средств для генерации и преобразования изображений налету ImageDataGenerator. Для всех моделей использовалось одинаковое количество эпох. Обучение и тестирование было реализовано с помощью обертки keras, которая позволяет использовать scikit-learn API с унифицированным интерфейсом для обучения, применения и проверки точности модели.

Решение задачи проходило в несколько этапов:

\begin{itemize}
  \item Реализация существующих моделей,
  \item Вычисление метрик каждой модели на каждом наборе данных,
  \item Анализ результатов
  \item Построение новой модели и подбор параметров
  \item Сравнение результатов новой модели и существующих
\end{itemize}

