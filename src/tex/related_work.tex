\section{Обзор существующих методов}

В области копьютерного зрения и машинного обучения лучший результат сейчас показывают нейронные сети. Этот подход и было решено применить для решения задачи сегментации снимков МРТ сердца.

\subsection{Простая сверточная сеть}
 
Простая модель состоит из нескольких сверточных блоков и нескольких обратных сверточных блоков. Каждый блок, в свою очередь, состоит из нескольких сверточных слоев и слоя max pooling. 

Каждый обратный блок состоит из 

\begin{enumerate}
	\item сверточного слоя с ядром 1х1 и шагом 1
	\item обратного сверточного слоя с ядром 3x3 и шагом 2
	\item сверточного слоя с ядром 1х1 и шагом 1
	\item слоя с обрезанием
\end{enumerate}

Такая архитектура сети была выбрана по нескольким причинам:

\begin{itemize}
	\item в сети используются только сверточные слои, чтобы
	\begin{enumerate}
		\item минимизировать количество параметров сети,
		\item ускорить работу сети,
	\end{enumerate}
	\item нормализация с помощью mvn-слоев позволяет улучшить сходимость сети,
	\item обратные сверточные слои нужны, чтобы получить из высокоуровневых признаков тепловую карту вероятностей принадлежности входных пикселей объекту в зависимости от контекста,
	\item несколько обратных слоев позволяют сгладить резкие переходы на выходной карте вероятностей и получить более точные предсказания.
 \end{itemize}

\subsection{U-Net}

\subsection{GridNet}
