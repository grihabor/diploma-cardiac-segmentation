\section{Обзор существующих методов}

В~области копьютерного зрения и~машинного обучения лучший результат сейчас показывают нейронные сети. Этот подход и~было решено применить для~решения задачи сегментации снимков МРТ сердца.

\subsection{Простая сверточная сеть}
 
Простая модель состоит из~нескольких сверточных блоков и~нескольких обратных сверточных блоков. Каждый блок, в~свою очередь, состоит из~нескольких сверточных слоев и~слоя max~pooling. 

Каждый сверточный блок состоит из~нескольких слоев, расположенных друг за~другом:

\begin{itemize}
  \item сверточный слой с~ядром~3x3
  \item mvn слой
  \item сверточный слой с~ядром~3x3
  \item mvn слой
  \item \dots
  \item maxpool слой с~ядром~2x2 и~шагом~2
\end{itemize}

Всего в~нейронной сети 3~таких блока и~1~блок без~maxpool слоя. Они идут подряд, и~каждый следующий на~выходе дает в~2~раза больше фильтров. Обозначим сверточный блок с~$k$~свертками и~$n$~фильтрами как~$conv(k,n)$.

Структура обратного блока, напротив, нелинейная:

\begin{enumerate}
	\item обратный сверточный слой с~ядром~3x3 и~шагом~2,
	\item сверточный слой с~ядром~1х1 и~шагом~1, который вычисляется от~предыдущего mvn~слоя,
	\item слой, который вычисляет среднее арифметическое из~первых двух слоев блока.
\end{enumerate}

Всего в нейронной сети 3~обратных блока. Количество фильтров в каждой свертке обратного блока равно количеству классов. Если же класса всего~2, то мы задаем на выходе 1~фильтр и~в~конце используем функцию активации sigmoid, вычисляя ошибку при этом с~помощью индекса Дайса. В данной работе во~всех случаях используется только 2~класса.

Таким образом, вся архитектура сети выглядит следующим образом:

\begin{itemize}
  \item Вход
  \item $conv(3,64)$
  \item $conv(4,128)$
\end{itemize}

Такая архитектура сети была выбрана неслучайно.

\begin{itemize}
	\item В~сети используются только сверточные слои, чтобы минимизировать количество параметров сети, и~ускорить работу сети.
	\item Нормализация с~помощью mvn~слоев позволяет улучшить сходимость сети.
	\item Обратные сверточные слои нужны, чтобы получить из~высокоуровневых признаков тепловую карту вероятностей принадлежности входных пикселей объекту в~зависимости от~контекста.
	\item Несколько обратных слоев позволяют сгладить резкие переходы на~выходной карте вероятностей и~получить более точные предсказания.
	\item В~обратных блоках учитываются выходы соответствующих mvn слоев прямых блоков, чтобы увеличить скорость обучения. Этот прием очень похож на~residual блоки в~\cite{resnet}.
 \end{itemize}

\subsection{U-Net}

\subsection{GridNet}
