\section{Обзор существующих методов}
Задача оценки степени дискомфорта стереоскопических изображений и видео по их параметрам рассматривается в литературе, однако лишь в небольшом количестве работ эксперименты были посвящены анализу искажений в стереопаре, характерных для съёмки стереофильмов. Например, в работе \cite{khaustova2015objective} рассматривается влияние геометрических несоответствий в стереопаре на уровень дискомфорта, но эксперименты проводились только с использованием стереоизображений. В работе \cite{dumic20173d} рассматривается 22 типа искажений, в том числе, геометрические и цветовые. Однако авторы не исследовали временное несоответствие между ракурсами (сдвиг во времени одного из ракурсов стереофильма относительно другого), а характер добавленных в эксперименте геометрических искажений редко встречается при съёмке реальных стереофильмов.