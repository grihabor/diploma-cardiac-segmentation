\section{Обзор существующих методов}

Существует много алгоритмов для решения задачи семантической сегментации или задачи поиска контура. Сюда входят, в том числе, методы активного контура~\cite{snakes}, сегментация с помощью кластеризации~\cite{clustering_segm} и нейросетевые методы~\cite{fcn},~\cite{unet},~\cite{gridnet}.

\subsection{Простая сверточная модель}

Простая сверточная модель~\cite{fcn_1_layer_upsample} была разработана на основе сверточных нейронных сетей для классификации. Отличие состоит в том, что в сетях для класификации последний слой softmax~\cite{classification_loss} преобразует высокоуровневые признаки, полученные последовательным применением сверток в вероятности принадлежности к классам, а~для~сегментации сначала используется upsampling слой, который преобразует высокоуровневые признаки в~изображение размером с~исходное, которое описывает, где именно находится объект. Каждый сверточный блок состоит из~нескольких слоев, расположенных друг за~другом, как описано в~таблице~\ref{tab:conv_block}.

\begin{table}[b]
  \begin{center}
    \begin{tabular}{ c }
      \hline
      сверточный слой с~ядром~3x3           \\ \hline
      mvn слой                              \\ \hline
      сверточный слой с~ядром~3x3           \\ \hline
      mvn слой                              \\ \hline
      \dots                                 \\ \hline
      maxpool слой с~ядром~2x2 и~шагом~2    \\ 
      \hline
    \end{tabular}
    \caption{Структура сверточного блока} \label{tab:conv_block}
  \end{center}
\end{table}

Такая структура часто используется в нейронных сетях, потому что позволяет ускорить сходимость и~уменьшить количество вычислений при~сохранении точности. Сверточные слои обеспечивают вычисление высокоуровневых признаков с помощью применения фильтров к входному изображению. Maxpool слои позволяют уменьшить сложность вычислений, и оставить самую важную информацию, полученную из сверточных слоев. Слой dropout помогает избежать переобучения~(\cite{dropout}). 

\subsection{Модель c обратной сверткой}
 
Модель с обратной сверткой~\cite{fcn} более сложная. Она состоит из~нескольких сверточных блоков и~нескольких обратных сверточных блоков. Каждый блок, в~свою очередь, состоит из~нескольких сверточных слоев с~функцией активации relu и~maxpool~слоя. 

Всего в~нейронной сети 3~сверточных блока и~1~блок без~maxpool слоя. Они идут подряд, и~каждый следующий на~выходе дает в~2~раза больше фильтров. 

Структура обратного блока, напротив, нелинейная:

\begin{itemize}
  \item обратный сверточный слой с~ядром~3x3 и~шагом~2,
  \item сверточный слой с~ядром~1х1 и~шагом~1, который вычисляется от~предыдущего mvn~слоя,
  \item слой, который вычисляет среднее арифметическое из~первых двух слоев блока.
\end{itemize}

Всего в нейронной сети 3~обратных блока. Количество фильтров в каждой свертке обратного блока равно количеству классов. Если же класса всего~2, то мы задаем на выходе 1~фильтр и~в~конце используем функцию активации sigmoid, вычисляя ошибку при этом с~помощью индекса Дайса. В данной работе во~всех случаях используется только 2~класса.

Обозначим сверточный блок с~$k$~свертками и~$n$~фильтрами как~$conv(k,n)$. Обозначим обратный сверточный слой, который получает на одном из входов $n$~фильтров как~$deconv(n)$. Обозначим слой dropout регуляризации с параметром $p$ как $dropout(p)$. Архитектура всей сети описана в таблице \ref{tab:fcn}.

\begin{table}
  \begin{center}
    \begin{tabular}{ c }
      \hline
      вход                              \\ \hline
      $conv(3,64)$                      \\ \hline
      $conv(4,128)$                     \\ \hline
      $conv(4,256)$                     \\ \hline
      $droupout(0.5)$                   \\ \hline
      $conv(4,512)$ без maxpool слоя    \\ \hline
      $droupout(0.5)$                   \\ \hline
      $deconv(512)$                     \\ \hline
      $deconv(256)$                     \\ \hline  
      $deconv(128)$                     \\ \hline      
      sigmoid активация                 \\
      \hline
    \end{tabular}
    \caption{Архитектура модели с обратной сверткой} \label{tab:fcn}
  \end{center}
\end{table}

Чтобы сеть сходилась устойчиво, после каждой свертки используется mvn слой~(\cite{batch_norm}). Обратные сверточные слои позволяют получить карту вероятностей принадлежности входных пикселей объекту в~зависимости от~контекста из~высокоуровневых признаков, вычисленных сверточными слоями. Каждый блок обратной свертки связан не только с предыдущим блоком обратной свертки, но и с соответствующим по размеру сверточным блоком, для того, чтобы обеспечить сходимость сети. Несколько обратных слоев позволяют сгладить резкие переходы на~выходной карте вероятностей и~получить более точные предсказания. В~обратных блоках учитываются выходы соответствующих mvn слоев прямых блоков, чтобы увеличить скорость обучения. Этот прием очень похож на~residual блоки в~\cite{resnet}.

\subsection{U-Net}

Архитектура U-Net \cite{unet} основана на идеях, перечисленных при описании более простых моделей. Здесь также используются прямые и обратные блоки. 

Каждый прямой блок модели U-Net состоит из 2-х сверточных слоев с ядром 3х3 и функцией активации relu. В обратных блоках используются простые upsampling слои для получения изображения из высокоуровневых сетей. Кроме того, в обратных блоках присутствует merge слой, который конкатенирует выход соответствущего по размеру прямого блока и выход предыдущего обратного блока. Такое решение позволяет получать быструю сходимость сети, так как при алгоритме обратного распространения ошибки, последняя не затухает, а сразу передается на сверточные блоки. Помимо этого, точность модели улучшается за счет увеличения количества параметров.   

\subsection{GridNet}

Модель Gridnet~\cite{gridnet} являтся развитием модели U-Net.

\subsection{DeepLab}

Большинство сверточных сетей использует комбинацию сверточных слоев и maxpool слоев. Из-за глубокой архитектуры это позволяет сверточным слоям учитывать большой контекст исходного изображения. Например, если сначала идет свертка с ядром 3x3, а затем maxpool с ядром 2х2, то следующая свертка с ядром 3x3 будет учитывать часть исходного изображения размером 7х7. С помощью использования большого контекста можно вовсе отказаться от полносвязных слоев в пользу сверточных, причем будет достаточно только слоев с ядром 3х3.

Альтернативой слою maxpool являются растянутые сверточные слои~\cite{dilated_conv}. Такие слои используют меньше входов, но охватывают большой контекст, игнорируя некоторые входы внутри ядра. Так, например, свертка с ядром 3х3 и растяжением 2 по сути является сверткой 5х5, в которой используется только 9 входов. Использование растянутых сверточных слоев позволяет улучшить уже существующие модели или создать новые модели, которые вовсе не используют maxpool слои.
