\section{Исследование и построение решения задачи}

\subsection{Наборы данных}

\paragraph{Sunnybrook}

Набор содержит снимки 45 пациентов. У разных пациентов, снимки которых присутствуют в наборе, разное состояние здоровья сердца: у некоторых заболеваний~нет, у~остальных — разные виды заболеваний. Для~обучения присутствует экспертная разметка эндокарда, эпикарда и~сосочковых мышц. Набор разделен на 3~части по~15~пациентов в~каждом: часть для обучения, для валидации и~для~тестирования.

\paragraph{LVSC}

Набор содержит снимки 200 пациентов. Набор разделен на 2 части пополам: часть для обучения и~часть для~валидации. У всех пациентов присутствует ишемическая болезнь сердца и коронарная недостаточность.

\paragraph{RVSC}

Набор состоит из 48 снимков пациентов с разными заболеваниями сердца. Данные разделены на 3 части: одна для обучения и две для тестирования. Экспертная разметка присутствует только для части для обучения.

\subsection{U-Net с растянутой сверткой}

Для построения собственного решения были проанализированы существующие подходы и нейронные сети для решения задачи семантической сегментации и выделения контура сердца. В этой работе предложена своя архитектура сети, решающая эту задачу.

За основу была взята модель U-Net, потому что она дает хороший результат и ее количество параметров гораздо меньше, чем у GridNet. Идея архитектуры в том, чтобы заменить последовательные свертки в U-Net на последовательные растянутые свертки.

