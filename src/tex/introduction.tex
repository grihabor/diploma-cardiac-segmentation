\section{Введение}
Несмотря на популярность стереокино, зрители часто испытывают дискомфорт при просмотре стереофильмов. Авторы \cite{zeri2015visual} исследовали реакцию зрителей на просмотр стереофильмов: после сеансов в кинотеатрах было опрошено 854 зрителя. В результате исследования были выделены группы симптомов, вызванных просмотром стереофильмов. Среди них самыми частыми были напряжение глаз, нечёткость зрения и жжение в глазах. Одной из основных причин плохого самочувствия зрителей является наличие искажений в стереопаре. Например, такие несоответствия, как поворот одной из камер относительно оптической оси или вертикальный сдвиг одной из камер относительно другой при съёмке, не встречаются в жизни и приводят к необходимости анализа невозможной ситуации. Даже высокобюджетные фильмы, такие, как «Хранитель времени» и «Пираты Карибского моря: На странных берегах», содержат сцены с геометрическими и цветовыми несоответствиями между ракурсами \cite{9report}. Самый надёжный способ избежать появления таких несоответствий – контролировать качество стереоскопического контента на этапе его производства. В настоящее время существуют алгоритмы, способные обнаружить различные несоответствия в стереопаре \cite{voronov2013methodology}, однако зная о наличии какого-либо несоответствия, нельзя точно сказать, насколько болезненным оно будет для зрителя, при этом исправление стоит денег и занимает время. Различные типы несоответствий, а также их интенсивность по-разному влияют на уровень дискомфорта \cite{rozhkova}.

У дискомфорта, возникающего у зрителей во время просмотра стереоскопического видео, есть много причин. Они включают в себя низкое качество контента (например, стереоскопические артефакты, неестественные искажения, вызванные 2D-3D конвертацией и др.), а также несоблюдение рекомендаций для комфортного просмотра стереоскопического контента (некачественное оборудование, нахождение вне допустимой комфотрной зоны во время просмотра). Головные боли и другие симптомы могут быть результатом высокой нагрузки на визуальную систему человека, которую вызывают неестественные искажения стереоскопического видео. Эта ситуация приводит к конфликту между аккомодацией и вергенцией, который широко обсуждался в литературе \cite{lambooij2009visual, terzic2017causes}. В данном исследовании исследуется влияние на степень дискомфорта зрителей артефактов, характерных для фильмов, произведенных методом стереоскопической съемки.