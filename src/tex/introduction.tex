\section{Введение}

Благодаря современным технологиям стало возможным обследование огромного количества людей в автоматическом режиме. Это, в свою очередь, позволяет диагностировать болезни на ранней стадии, улучшать профилактику заболеваний, сохранять жизнь многим людям.

В современном мире можно очень легко и быстро собрать детализированную информацию о пациенте, проанализировать ее и быстро среагировать на любое отклонение от нормы. Основные способы получения информации о пациенте включают: рентгенографию, компьютерную томографию, маммографию, магнитно-резонансная томография (МРТ), ультразвуковое исследование (УЗИ), а также различные виды анализов. Такое разнообразие показателей позволяет сильно упростить исследования с помощью автоматизации процесса.

Технологии анализа больших данных и машинного обучения позволяют построить эффективные модели, которые смогут решать самые разные задачи: по результатам анализов можно предсказать вероятности присутствия той или иной болезни, а по снимкам рентгена или МРТ можно автоматически выявить отклонения размеров органов или их положения от нормы. Автоматизация решения этих задач позволяет в разы ускорить процесс диагностирования заболеваний, а также даёт врачам возможность заниматься более сложными случаями и спасать больше людей.

Целью этой работы является построение алгорима, решающего задачу сегментации структур сердца на снимках МРТ сердца. В частности, в данной работе рассматриваются эпикард и эндокард левого и правого желудочков. Результат решения такой задачи позволяет определить точное положение сердца относительно других органов и~тканей, оценить размер сердца в~заданной фазе и~даже построить объемную модель сердца пациента.

При решении задачи сегментации изображений МРТ сердца возникает много проблем, специфичных этой области. Ниже перечислены некоторые из~них:

\begin{itemize}
  \item распределения интенсивности пикселей структур сердца и окружающих его тканей пересекаются;
  \item форма эндокарда и эпикарда может сильно отличаться в зависимости от разреза и фазы;
  \item данные распределены неравномерно: пикселей фона гораздо больше, чем пикселей объекта;
  \item неточная, расплывчатая информация о краях объекта, особенно на крайних срезах;
  \item большие отличия в снимках МРТ в зависимости от разных сканеров, учреждений и людей;
  \item шумы на снимках МРТ.
\end{itemize}

Чтобы решить эти проблемы существует много подходов. В~области копьютерного зрения и~машинного обучения лучший результат при обратке изображений сейчас показывают нейронные сети. Этот подход и~было решено применить для~решения задачи сегментации снимков МРТ сердца. Кроме того, нейросетевые модели позволяют решить задачу полностью автоматически, то есть без подсказок эксперта.
