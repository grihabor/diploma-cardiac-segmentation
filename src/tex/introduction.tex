\section{Введение}
Медицина - широкий раздел биологии, который занимается спектром задач, связанным с диагностикой, лечением и профилактикой заболеваний, а также с сохранением и укреплением здоровья и трудоспособности людей. 

Благодаря современным технологиям стало возможным обследование огромного количества людей в автоматическом режиме. Это, в свою очередь, позволило диагностировать болезни на ранней стадии, улучшить профилактику заболеваний, сохранить жизнь многим людям.

В современном мире можно очень легко и быстро собрать детализированную информацию о пациенте, проанализировать ее и быстро среагировать на любое отклонение от нормы. Основные способы получения информации о пациенте включают: рентгенографию, компьютерную томографию, маммографию, магнитно-резонансная томография (МРТ), ультразвуковое исследование (УЗИ), а также различные виды анализов. Такое разнообразие показателей позволяет сильно упростить исследования с помощью автоматизации процесса.

Людей, которые профессионально занимаются медициной, очень мало, потому что обучение по этой специальности занимает очень много времени. По этой причине особенно важно максимально освободить этих людей от рутинной работы, не говоря уже о том, что человек легко может сделать ошибку.

Технологии анализа больших данных и машинного обучения позволяют построить эффективные модели, которые смогут решать самые разные задачи: по результатам анализов можно предсказать вероятности присутствия той или иной болезни, а по снимкам рентгена или мрт можно автоматически выявить отклонения размеров органов или их положения от нормы. Автоматизация решения этих задач позволяет в разы ускорить процесс диагностирования заболеваний, а также даёт врачам возможность заниматься более сложными случаями и спасать больше людей.

Целью этой работы является построение алгорима, решающего задачу сегментации снимков МРТ сердца. Результат решения такой задачи позволяет определить точное положение сердца относительно других органов и тканей, оценить размер сердца в заданной фазе и даже построить объемную модель сердца пациента.

Среди методов сегментации выделяют автоматические и полуавтоматические. 

В~области копьютерного зрения и~машинного обучения лучший результат сейчас показывают нейронные сети. Этот подход и~было решено применить для~решения задачи сегментации снимков МРТ сердца.

Основные проблемы, возникающие при решении задачи сегментации изображений МРТ сердца:

\begin{itemize}
  \item распределения интенсивности пикселей структур сердца и окружающих его тканей пересекаются;
  \item формы эндокарда и эпикарда могут сильно отличаться в зависимости от разреза и фазы;
  \item данные распределены неравномерно: пикселей фона гораздо больше, чем пикселей объекта;
  \item неточная, расплывчатая информация о краях объекта, особенно на крайних срезах;
  \item большие отличия в снимках МРТ в зависимости от разных сканеров, учреждений и людей;
  \item шумы на снимках МРТ.
\end{itemize}


