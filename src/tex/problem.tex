\section{Постановка задачи}

Целью данной работы является построение алгоритма, который решает задачу сегментации снимков МРТ сердца. Задача сегментации заключается в том, чтобы определить, какие части изображения принадлежат целевому объекту, а какие нет. 

\subsection{Входные данные}

Каждый набор данных разделен на 3 части: 

\begin{itemize}
  \item данные для обучения
  \item данные для валидации
  \item данные для тестирования
\end{itemize}

Данные для обучения и валидации — это пары $(\hat{X}_{train},\hat{Y}_{train})$ и $(\hat{X}_{val},\hat{Y}_{val})$, где $\hat{X}_{train} = \{\hat{x}_{train}^{i}\}, i\in{}\overline{1,n_{train}}$ и $\hat{X}_{val} = \{\hat{x}_{val}^{i}\}, i\in{}\overline{1,n_{val}}$ соответственно — 2 упорядоченных набора матриц, каждый элемент каждой из которых представляет собой интенсивность соответствующего участка на снимке МРТ сердца, $\hat{Y}_{train}$ и $\hat{Y}_{val}$ соответственно — 2 упорядоченных множества пар $(x_{i},y_{i}), i = \overline{1,m}$, определяющие 2 соответствующих \mbox{$m$-угольника} — выделенные экспертом контуры сердца.

Данные для тестирования — упорядоченный набор матриц $\hat{X}_{test} = \{\hat{x}_{test}^{i}\}, i\in{}\overline{1,n_{test}}$, каждый элемент каждой из которых представляет собой интенсивность соответствующего пикселя на снимке МРТ сердца, для которого необходимо предсказать контур искомого объекта.

\subsection{Формальная постановка задачи}

Чтобы использовать нейронную сеть для решения задачи, преобразуем каждый набор $\hat{X},\hat{X}\in{}\{\hat{X}_{train},\hat{X}_{val},\hat{X}_{test}\}$ в соответствующий новый упорядоченный набор $X,X\in{}\{X_{train},X_{val},X_{test}\}$, каждый элемент которого — матрица $R^{n\times{}n}$, каждый элемент которой, в свою очередь, представляет собой нормализованную интенсивность соответствующего участка на снимке МРТ сердца~\eqref{eq:input_normalized}. 

\begin{equation}
\label{eq:input_squared}
(X^{2})_{i,j}=((X)_{i,j})^{2}
\end{equation}

\begin{equation}
\label{eq:input_expected_value}
E[X]=\frac{
  \sum_{i,j}X
}{
  n^{2}
}
\end{equation} 

\begin{equation}
\label{eq:input_normalized}
X = \frac{
  \hat{X} - E[\hat{X}]
}{\sqrt{
  E[\hat{X}^{2}] - (E[\hat{X}])^2
}}
\end{equation}

Преобразуем каждый набор $\hat{Y},\hat{Y}\in{}\{\hat{Y}_{train},\hat{Y}_{val}\}$ в соответствующий упорядоченный набор $Y,Y\in{}\{Y_{train},Y_{val}\}$, каждый элемент которого — матрица $R^{n\times{}n}$. Каждый элемент последней, в свою очередь, является числом $1$ или $0$, в зависимости от того, принадлежит ли соответствующий пиксель заданному \mbox{$m$-угольнику} или нет.

Таким образом, исходная задача сводится к задаче бинарной классификации каждого пикселя входного изображения — либо пиксель принадлежит объекту, либо нет. На вход алгоритму подаются нормализованные матрицы снимков МРТ, а на выходе алгоритма получается матрица, каждый элемент которой показывает вероятность принадлежности соответствующего пикселя объекту. Выход алгоритма затем сравнивается с экспертной разметкой с помощью различных метрик.

\subsection{Метрики}

Чтобы оценить качество сегментации, введем метрики оценки качества сегментации. Результат сегментации можно представить либо как область пикселей, либо как контур вокруг этой области. 

Индекс Дайса~\eqref{eq:dice_index} и мера Жаккара~\eqref{eq:jaccard_index} позволяют сравнить 2 области:

\begin{equation}\label{eq:dice_index}
  Dice(U,V) = \frac{2|U\cap{}V|}{|U| + |V|}
\end{equation}
\begin{equation}\label{eq:jaccard_index}
  J(U,V) = \frac{|U\cap{}V|}{|U\cup{}V|}
\end{equation}

Метрика Хаусдорфа~\eqref{eq:hausdorff_distance} позволяет оценить схожесть контуров:

\begin{equation}\label{eq:hausdorff_distance}
  d_{H}(A,B)=\max\left\{\sup_{a\in{}A}\inf_{b\in{}B}d(a,b),\sup_{b\in{}B}\inf_{a\in{}A}d(a,b)\right\}
\end{equation}

\clearpage
Чтобы посчитать меру Жаккара или индекс Дайса, необходимо 
предварительно преобразовать экспертную разметку, а чтобы 
вычислить метрику Хаусдорфа, необходимо преобразовать матрицу 
вероятностей в контур. Пусть на входе алгоритму подается 
матрица $X$, на выходе алгоритм выдает матрицу $U$, которой
соответствует контур $\hat{U}$, тогда результаты будут 
вычисляться по формулам~\eqref{eq:result_dice},~\eqref{eq:result_jaccard},~\eqref{eq:result_hausdorff}. В данной работе для обучения, тестирования и сравнения моделей использовался индекс Дайса~\eqref{eq:dice_index}.

\begin{equation}\label{eq:result_dice}
  Result_{Dice} = Dice(Y, U)
\end{equation}
\begin{equation}\label{eq:result_jaccard}
  Result_{Jaccard} = J(Y, U)
\end{equation}
\begin{equation}\label{eq:result_hausdorff}
  Result_{Hausdorff} = d_{H}(\hat{Y}, \hat{U})
\end{equation}
