\section{Постановка задачи}

Целью данной работы является построение алгоритма, который решает задачу сегментации снимков МРТ сердца. Задача сегментации заключается в том, чтобы определить, какие части изображения принадлежат целевому объекту, а какие нет. 

\subsection{Входные данные}

Каждый набор данных разделен на 3 части: 

\begin{itemize}
  \item данные для обучения
  \item данные для валидации
  \item данные для тестирования
\end{itemize}

Данные для обучения и валидации — это пары $(\hat{X}_{train},\hat{Y}_{train})$ и $(\hat{X}_{val},\hat{Y}_{val})$, где

  $\hat{X}_{train} = \{\hat{x}_{train,i}\},i\in{}\overline{1,n_{train}}$ и $\hat{X}_{val} = \{\hat{x}_{val,i}\},i\in{}\overline{1,n_{val}}$ соответственно — 2 упорядоченных набора матриц, каждый элемент каждой из которых представляет собой интенсивность соответствующего участка на снимке МРТ сердца,
  
  $\hat{Y}_{train}$ и $\hat{Y}_{val}$ соответственно — 2 упорядоченных множеств пар $(x_{i},y_{i}), i = \overline{1,m}$, определяющее \mbox{$m$-угольник} — выделенный экспертом контур сердца.

Данные для тестирования — только матрица $\hat{X}$.

\subsection{Формальная постановка задачи}

Для решения задачи с помощью нейросетей преобразуем пару $(\hat{X},\hat{Y})$ к паре $(X,Y)$, где

  $X$ — матрица $R^{n*n}$, каждый элемент которой представляет собой нормализованную интенсивность соответствующего участка на снимке МРТ сердца,

  $Y$ — матрица $R^{n*n}$, каждый элемент которой соответствует вероятности принадлежности соответствующего пикселя объекту. 

Таким образом, исходная задача сводится к задаче бинарной классификации каждого пикселя входного изображения — либо пиксель принадлежит объекту, либо нет.

\subsection{Метрики}

Чтобы оценить качество сегментации, введем метрики оценки качества сегментации. Результат сегментации можно представить либо как область пикселей, либо как контур вокруг этой области. 

Индекс Дайса (Dice index) позволяет сравнить 2 области:

$Dice(A,B) = \frac{2|A\cap{}B}{|A| + |B|}$

Индекс Джаккарда (Jaccard index) — еще одна метрика для сравнения областей:

$J(A,B) = \frac{|A\cap{}B|}{|A\cup{}B}$

Метрика Хаусдорфа (Hausdorff distance) позволяет оценить схожесть контуров:

$\mathrm{d}_{H}(A,B)=\max\left\{\sup_{a\in{}A}\inf_{b\in{}B}\mathrm{d}(a,b),\sup_{b\in{}B}\inf_{a\in{}A}\mathrm{d}(a,b)\right\}$

Чтобы посчитать индекс Джаккарда или индекс Дайса, необходимо предварительно преобразовать экспертную разметку


