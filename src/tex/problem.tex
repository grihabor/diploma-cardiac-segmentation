\section{Постановка задачи}

Целью данной работы является построение алгоритма, который решает задачу сегментации снимков МРТ сердца. Задача сегментации заключается в том, чтобы определить, какие части изображения принадлежат целевому объекту, а какие нет. 

Формальная постановка задачи

Данные имеют следующий вид:

вход 

Матрица $R^{n*n}$, где каждый элемент представляет собой интенсивность соответствующего участка на снимке МРТ сердца

выход

Упорядоченное множество пар $(x_i,y_i), i = \overline{1,m}$, описывающее выделенный экспертом контур сердца.

Преобразование данных

Для решения задачи данные преобразовывались в следующий вид:

вход

Матрица $R^{n*n}$, где каждый элемент представляет собой нормализованную интенсивность соответствующего участка на снимке МРТ сердца.

выход

Матрица $R^{n*n}$, где каждый элемент соответствует вероятности принадлежности соответствующего пикселя объекту. 

Таким образом, исходная задача сводится к задаче бинарной классификации каждого пикселя входного изображения — либо пиксель принадлежит объекту, либо нет.
