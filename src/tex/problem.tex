\section{Постановка задачи}

Целью данной работы является построение алгоритма, который решает задачу сегментации снимков МРТ сердца. Задача сегментации заключается в том, чтобы определить, какие части изображения принадлежат целевому объекту, а какие нет. 

Формальная постановка задачи

Данные имеют следующий вид:

вход 

Матрица $R^{n*n}$, где каждый элемент представляет собой интенсивность соответствующего участка на снимке МРТ сердца

выход

Упорядоченное множество пар $(x_i,y_i), i = \overline{1,m}$, описывающее выделенный экспертом контур сердца.

Преобразование данных

Для решения задачи данные преобразовывались в следующий вид:

вход

Матрица $R^{n*n}$, где каждый элемент представляет собой нормализованную интенсивность соответствующего участка на снимке МРТ сердца.

выход

Матрица $R^{n*n}$, где каждый элемент соответствует вероятности принадлежности соответствующего пикселя объекту. 

Таким образом, исходная задача сводится к задаче бинарной классификации каждого пикселя входного изображения — либо пиксель принадлежит объекту, либо нет.

\subsection{Метрики}

Чтобы оценить качество сегментации, введем метрики оценки качества сегментации. Результат сегментации можно представить либо как область пикселей, либо как контур вокруг этой области. 

Метрика Дайса (Dice index) позволяет сравнить 2 области:

$Dice(X,Y) = \frac{2|X\cap{}Y}{|X| + |Y|}$

Индекс Джаккарда (Jaccard index) — еще одна метрика для сравнения областей:

$J(X,Y) = \frac{|X\cap{}Y|}{|X\cup{}Y}$

Метрика Хаусдорфа (Hausdorff distance) позволяет оценить схожесть контуров:

$\mathrm{d}_{H}(A,B)=\max\left\{\sup_{a\in{}A}\inf_{b\in{}B}\mathrm{d}(a,b),\sup_{b\in{}B}\inf_{a\in{}A}\mathrm{d}(a,b)\right\}$


