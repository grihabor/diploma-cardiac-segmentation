\section{Постановка задачи}
Целью данной работы является разработка метода прогнозирования уровня дискомфорта, который зритель испытает при просмотре определенного фрагмента стереоскопического видео. Разрабатываемый метод будет использовать информацию о техническом качестве стереофильма.

В данной работе будут использованы следующие понятия:
\begin{itemize}
\item{Стереофильм — }
\item{Стереопара — к}
\item{Ракурс (левый или правый) — }
\item{Усталость, дискомфорт, головная боль — термины используются для обозначения негативного самочувствия зрителя во время просмотра стереофильма.}
\item{Параллакс — изменение видимого положения объекта относительно удаленного фона в зависимости от положения наблюдателя.}
\end{itemize}

Автоматическая система определения уровня усталости зрителя, вызванной просмотром стереофильма, должна по данным качества стереофильма давать оценку степени дискомфорта, который будет испытывать зритель при просмотре этого стереофильма. Для получения необходимых данных качества стереовидео будут использоваться автоматические средства оценки качества стереофильмов, разработанные в рамках проекта VQMT3D \cite{vqmt3d} лаборатории <<Компьютерной графики и мультимедиа>> МГУ. Используемые средства позволяют измерить величину несоответствий в стереопаре, которые возникают в процессе съемки стереофильма. В данной работе будет проводится исследование влияния на усталость четырех следующих видов несоответствий: расхождение ракурсов по цвету, масштабу, поворот одного из ракурсов относительно другого и несоответствие по времени. Для учета отдельных особенностей кадров стереофильма при анализе система так же будет использовать данные о величине параллакса, яркости кадра, информацию об интенсивности движения в сцене стереофильма. 