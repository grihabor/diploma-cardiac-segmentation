\documentclass[14pt, a4paper]{extarticle}
\usepackage[utf8]{inputenc}
\usepackage[russian]{babel}
\usepackage{graphicx}
\usepackage{float}
\usepackage{caption}
\usepackage{subcaption}
\usepackage{url}
\usepackage{multirow}
\usepackage{mathtools}
\usepackage{amsmath}
\usepackage[left=2cm,right=2cm,bottom=3cm,top=2cm]{geometry}
\usepackage{multirow}

\DeclarePairedDelimiter{\abs}{\lvert}{\rvert}

\linespread{1.25}

\begin{document}

\thispagestyle{empty}

\begin{center}
\ \vspace{-3cm}

\includegraphics[width=0.5\textwidth]{img/msu.eps}\\
{Московский государственный университет имени М.В.~Ломоносова}\\
Факультет вычислительной математики и кибернетики\\
Кафедра интеллектуальных информационных технологий

\vspace{5cm}

{\Large Анциферова~Анастасия~Всеволодовна}

\vspace{1cm}

{\Large\bfseries
Прогнозирование степени дискомфорта зрителей при просмотре стереоскопического фильма по его техническому качеству\\}

\vspace{1cm}

{\large МАГИСТЕРСКАЯ ДИССЕРТАЦИЯ}
\end{center}

\vfill

\begin{flushright}
  \textbf{Научный руководитель:}\\
  к.ф.-м.н., старший научный сотрудник\\
  Д.С.~Ватолин
\end{flushright}

\vfill

\begin{center}
Москва, 2018
\end{center}

\enlargethispage{4\baselineskip}

%\newpage
%\noindent 
%\textbf{Прогнозирование степени дискомфорта зрителей при просмотре стереоскопического фильма по его техническому качеству}\\
%Анциферова Анастасия\\
%В настоящее время большое количество фильмов производится в стереоскопическом формате. Несмотря на развитие технологии стереосъёмки, даже в высокобюджетных фильмах встречаются искажения, которые вызывают головную боль и дискомфорт у зрителей во время просмотра. Существующие методы автоматического контроля качества стереофильмов способны обнаружить такие искажения, однако оценка дискомфорта, вызванного отдельными искажениями, будучи сложной задачей, вызывает трудности. Данная работа посвящена исследованию влияния геометрических, цветовых и временных несоответствий между ракурсами стереофильма на уровень дискомфорта зрителя. В серии экспериментов было проведено анкетирование 302 человек по результатам просмотра фрагментов художественных стереофильмов с искусственно добавленными дозированными искажениями. Анкета состояла из вопросов о степени визуального дискомфорта, причиняемого каждым из продемонстрированных фрагментов, а также о поле и возрасте участника. Анализ полученных данных выявил зависимость между интенсивностью движения в сцене и степенью дискомфорта, причиняемой фрагментами с временным несоответствием.

%\noindent 
%\\[1mm]
%\textbf{Forecasting of viewers' discomfort degree when watching stereoscopic movie by its technical quality}\\
%Antsiferova Anastasiia\\
%Nowadays a large number of movies are produced in stereoscopic format. Despite stereo technology improvement, stereoscopic artifacts that cause headache and discomfort in viewers can still be found even in high-budget films. Existing automatic quality control algorithms are able to detect such distortions, however, they do not take into account a viewer's subjective perception of different artifacts. In this research the influence of geometric, color and temporal discrepancies in the stereo pair on a viewer's discomfort is explored. A series of experiments with passive stereo cinema technology was conducted. 60 video sequences from stereoscopic movies with artificially added distortions were demonstrated to the audience. 302 subjects took part in the experiments. During the analysis of the obtained data, the dependencies between the degree of spectators' discomfort and the intensity of distortions were revealed.

\clearpage
\newpage

\renewcommand{\contentsname}{Содержание}
\tableofcontents
\clearpage
\newpage

\section{Введение}
Несмотря на популярность стереокино, зрители часто испытывают дискомфорт при просмотре стереофильмов. Авторы \cite{zeri2015visual} исследовали реакцию зрителей на просмотр стереофильмов: после сеансов в кинотеатрах было опрошено 854 зрителя. В результате исследования были выделены группы симптомов, вызванных просмотром стереофильмов. Среди них самыми частыми были напряжение глаз, нечёткость зрения и жжение в глазах. Одной из основных причин плохого самочувствия зрителей является наличие искажений в стереопаре. Например, такие несоответствия, как поворот одной из камер относительно оптической оси или вертикальный сдвиг одной из камер относительно другой при съёмке, не встречаются в жизни и приводят к необходимости анализа невозможной ситуации. Даже высокобюджетные фильмы, такие, как «Хранитель времени» и «Пираты Карибского моря: На странных берегах», содержат сцены с геометрическими и цветовыми несоответствиями между ракурсами \cite{9report}. Самый надёжный способ избежать появления таких несоответствий – контролировать качество стереоскопического контента на этапе его производства. В настоящее время существуют алгоритмы, способные обнаружить различные несоответствия в стереопаре \cite{voronov2013methodology}, однако зная о наличии какого-либо несоответствия, нельзя точно сказать, насколько болезненным оно будет для зрителя, при этом исправление стоит денег и занимает время. Различные типы несоответствий, а также их интенсивность по-разному влияют на уровень дискомфорта \cite{rozhkova}.

У дискомфорта, возникающего у зрителей во время просмотра стереоскопического видео, есть много причин. Они включают в себя низкое качество контента (например, стереоскопические артефакты, неестественные искажения, вызванные 2D-3D конвертацией и др.), а также несоблюдение рекомендаций для комфортного просмотра стереоскопического контента (некачественное оборудование, нахождение вне допустимой комфотрной зоны во время просмотра). Головные боли и другие симптомы могут быть результатом высокой нагрузки на визуальную систему человека, которую вызывают неестественные искажения стереоскопического видео. Эта ситуация приводит к конфликту между аккомодацией и вергенцией, который широко обсуждался в литературе \cite{lambooij2009visual, terzic2017causes}. В данном исследовании исследуется влияние на степень дискомфорта зрителей артефактов, характерных для фильмов, произведенных методом стереоскопической съемки.
\clearpage
\newpage
\section{Постановка задачи}
Целью данной работы является разработка метода прогнозирования уровня дискомфорта, который зритель испытает при просмотре определенного фрагмента стереоскопического видео. Разрабатываемый метод будет использовать информацию о техническом качестве стереофильма.

В данной работе будут использованы следующие понятия:
\begin{itemize}
\item{Стереофильм — }
\item{Стереопара — к}
\item{Ракурс (левый или правый) — }
\item{Усталость, дискомфорт, головная боль — термины используются для обозначения негативного самочувствия зрителя во время просмотра стереофильма.}
\item{Параллакс — изменение видимого положения объекта относительно удаленного фона в зависимости от положения наблюдателя.}
\end{itemize}

Автоматическая система определения уровня усталости зрителя, вызванной просмотром стереофильма, должна по данным качества стереофильма давать оценку степени дискомфорта, который будет испытывать зритель при просмотре этого стереофильма. Для получения необходимых данных качества стереовидео будут использоваться автоматические средства оценки качества стереофильмов, разработанные в рамках проекта VQMT3D \cite{vqmt3d} лаборатории <<Компьютерной графики и мультимедиа>> МГУ. Используемые средства позволяют измерить величину несоответствий в стереопаре, которые возникают в процессе съемки стереофильма. В данной работе будет проводится исследование влияния на усталость четырех следующих видов несоответствий: расхождение ракурсов по цвету, масштабу, поворот одного из ракурсов относительно другого и несоответствие по времени. Для учета отдельных особенностей кадров стереофильма при анализе система так же будет использовать данные о величине параллакса, яркости кадра, информацию об интенсивности движения в сцене стереофильма. 
\clearpage
\newpage
\section{Обзор существующих методов}

Существует много алгоритмов для решения задачи семантической сегментации или задачи поиска контура. Сюда входят, в том числе, методы активного контура~\cite{snakes}, сегментация с помощью кластеризации~\cite{clustering_segm} и нейросетевые методы~\cite{fcn},~\cite{unet},~\cite{gridnet}.

\subsection{Простая сверточная сеть}
 
Простая модель~\cite{fcn} состоит из~нескольких сверточных блоков и~нескольких обратных сверточных блоков. Каждый блок, в~свою очередь, состоит из~нескольких сверточных слоев и~слоя max~pooling. 

Каждый сверточный блок состоит из~нескольких слоев, расположенных друг за~другом:

\begin{itemize}
  \item сверточный слой с~ядром~3x3
  \item mvn слой
  \item сверточный слой с~ядром~3x3
  \item mvn слой
  \item \dots
  \item maxpool слой с~ядром~2x2 и~шагом~2
\end{itemize}

Всего в~нейронной сети 3~таких блока и~1~блок без~maxpool слоя. Они идут подряд, и~каждый следующий на~выходе дает в~2~раза больше фильтров. 

Структура обратного блока, напротив, нелинейная:

\begin{enumerate}
  \item обратный сверточный слой с~ядром~3x3 и~шагом~2,
  \item сверточный слой с~ядром~1х1 и~шагом~1, который вычисляется от~предыдущего mvn~слоя,
  \item слой, который вычисляет среднее арифметическое из~первых двух слоев блока.
\end{enumerate}

Всего в нейронной сети 3~обратных блока. Количество фильтров в каждой свертке обратного блока равно количеству классов. Если же класса всего~2, то мы задаем на выходе 1~фильтр и~в~конце используем функцию активации sigmoid, вычисляя ошибку при этом с~помощью индекса Дайса. В данной работе во~всех случаях используется только 2~класса.

Обозначим сверточный блок с~$k$~свертками и~$n$~фильтрами как~$conv(k,n)$. Обозначим обратный сверточный слой, который получает на одном из входов $n$~фильтров как~$deconv(n)$. Обозначим слой dropout регуляризации с параметром $p$ как $dropout(p)$. Таким образом, вся архитектура сети выглядит следующим образом:

\begin{itemize}
  \item Вход
  \item $conv(3,64)$
  \item $conv(4,128)$
  \item $conv(4,256)$
  \item $droupout(0.5)$
  \item $conv(4,512)$ без maxpool слоя
  \item $droupout(0.5)$
  \item $deconv(512)$
  \item $deconv(256)$
  \item $deconv(128)$
  \item $sigmoid$ активация
\end{itemize}

Такая архитектура сети была выбрана неслучайно. Сверточные слои обеспечивают вычисление высокоуровневых признаков с помощью применения фильтров к входному изображению. Maxpool слои позволяют уменьшить сложность вычислений, и оставить самую важную информацию, полученную из сверточных слоев. Чтобы сеть сходилась устойчиво, после каждой свертки используется mvn слой~(\cite{batch_norm}). Слой dropout помогает избежать переобучения~(\cite{dropout}). Обратные сверточные слои позволяют получить карту вероятностей принадлежности входных пикселей объекту в~зависимости от~контекста из~высокоуровневых признаков, вычисленных сверточными слоями. Каждый блок обратной свертки связан не только с предыдущим блоком обратной свертки, но и с соответствующим по размеру сверточным блоком, для того, чтобы обеспечить сходимость сети. Несколько обратных слоев позволяют сгладить резкие переходы на~выходной карте вероятностей и~получить более точные предсказания. В~обратных блоках учитываются выходы соответствующих mvn слоев прямых блоков, чтобы увеличить скорость обучения. Этот прием очень похож на~residual блоки в~\cite{resnet}.

\subsection{U-Net}

Архитектура U-Net более сложная. 

\subsection{GridNet}

\clearpage
\newpage
\section{Исследование и построение решения задачи}

\subsection{Наборы данных}

\paragraph{Sunnybrook}

Набор содержит снимки 45 пациентов. У разных пациентов, снимки которых присутствуют в наборе, разное состояние здоровья сердца: у некоторых заболеваний~нет, у~остальных — разные виды заболеваний. Для~обучения присутствует экспертная разметка эндокарда, эпикарда и~сосочковых мышц. Набор разделен на 3~части по~15~пациентов в~каждом: часть для обучения, для валидации и~для~тестирования.

\paragraph{LVSC}

Набор содержит снимки 200 пациентов. Набор разделен на 2 части пополам: часть для обучения и~часть для~валидации. У всех пациентов присутствует ишемическая болезнь сердца и коронарная недостаточность.

\paragraph{RVSC}

Набор состоит из 48 снимков пациентов с разными заболеваниями сердца. Данные разделены на 3 части: одна для обучения и две для тестирования. Экспертная разметка присутствует только для части для обучения.

\subsection{U-Net с растянутой сверткой}

Для построения собственного решения были проанализированы существующие подходы и нейронные сети для решения задачи семантической сегментации и выделения контура сердца. В этой работе предложена своя архитектура сети, решающая эту задачу.

За основу была взята модель U-Net, потому что она дает хороший результат и ее количество параметров гораздо меньше, чем у GridNet. Идея архитектуры в том, чтобы заменить последовательные свертки в U-Net на последовательные растянутые свертки. В оригинальном U-Net в каждом блоке присутствует 2~сверточных слоя с~ядром~3x3 и~затем maxpool~слой. В~предложенной модели также используется 2~сверточных слоя, но второй из них в каждом блоке вычисляется с~растяжением~2, то~есть~эффективное ядро свертки будет 5x5. Размер контекста входного изображения в U-Net 140x140~\eqref{eq:unet_influence}, а~размер контекста предложенной архитектуры — 202x202~\eqref{eq:dilated_unet_influence}. Это существенно уменьшает возможность появления артефактов и ошибок второго рода.

\begin{equation} \label{eq:unet_influence}
1-3-5:10-12-14:28-30-32:64-66-68:136-138-140
\end{equation}
\begin{equation} \label{eq:dilated_unet_influence}
1-5-7:14-18-20:40-44-46:92-96-98:196-200-202
\end{equation}



\clearpage
\newpage
\section{Описание практической части}

Проект был~реализован на~языке Python~3. Для реализации моделей использовался фреймворк tensorflow и~библиотекa keras. Наборы данных были получены в формате DICOM, поэтому для распаковки данных использовалась соответствующая библиотека dicom. Обработка данных в памяти осуществлялось с помощью библиотеки numpy. Для работы с изображениями использовалась библиотека OpenCV. Аугментация данных реализована с помощью встроенных в~keras средств для генерации и преобразования изображений налету ImageDataGenerator. Для всех моделей использовалось одинаковое количество эпох. Обучение и тестирование было реализовано с помощью обертки keras, которая позволяет использовать scikit-learn API с унифицированным интерфейсом для обучения, применения и проверки точности модели.

Решение задачи проходило в несколько этапов:

\begin{itemize}
  \item Реализация существующих моделей,
  \item Вычисление метрик каждой модели на каждом наборе данных,
  \item Анализ результатов
  \item Построение новой модели и подбор параметров
  \item Сравнение результатов новой модели и существующих
\end{itemize}


\clearpage
\newpage
\section{Заключение}

\begin{table}[b]
  \begin{center}
    \caption{Результаты на наборе данных LVSC} \label{tab:lvsc_results}
    \begin{tabular}{ |*{5}{c|} }
      \hline
      \multirow{2}{*}{Метод}      & \multicolumn{2}{c|}{Индекс Дайса} \\ \cline{2-5}
                                  & Endocardium & Epicardium          \\ \hline
      \hline
      FCN                         & 0.92        & 0.96                \\ \hline
      U-Net                       &             &                     \\ \hline
      GridNet                     &             &                     \\ \hline
      U-Net с~растянутой сверткой &             &                     \\ 
      \hline
    \end{tabular}
  \end{center}
\end{table}


\clearpage
\newpage

\renewcommand{\bibname}{Список литературы}
\addcontentsline{toc}{section}{\bibname}

\bibliographystyle{gost71s}
\bibliography{references}{}

\end{document} 